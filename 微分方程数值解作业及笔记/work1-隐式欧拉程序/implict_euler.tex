\documentclass{ctexart}
% \usepackage[UTF-8]{ctex}
\usepackage{amsmath}
\usepackage{booktabs}
\usepackage{multirow}
\usepackage{tabularx}

\usepackage{booktabs,multirow,longtable}

\usepackage{listings}
\usepackage{listings}
\usepackage{color}

%导言区插入下面三行
\usepackage{graphicx} %插入图片的宏包
\usepackage{float} %设置图片浮动位置的宏包
\usepackage{subfigure} %插入多图时用子图显示的宏包


\definecolor{dkgreen}{rgb}{0,0.6,0}
\definecolor{gray}{rgb}{0.5,0.5,0.5}
\definecolor{mauve}{rgb}{0.58,0,0.82}

\lstset{frame=tb,
  language=Python,
  aboveskip=3mm,
  belowskip=3mm,
  showstringspaces=false,
  columns=flexible,
  basicstyle={\small\ttfamily},
  numbers=left,
  numberstyle=\tiny\color{gray},
  keywordstyle=\color{blue},
  commentstyle=\color{dkgreen},
  stringstyle=\color{mauve},
  breaklines=true,
  breakatwhitespace=true,
  tabsize=3
}


\title{作业1:隐式欧拉程序}
\author{谢文进}
\date{\today}
\begin{document}
\maketitle
\section{隐式欧拉方法}
求解如下常微分方程:
\begin{equation} \label{eq_arry}%要调用宏包amsmath
    \begin{cases}
        \frac{du}{dt}=-\frac{u}{t},1\leq t\leq 2\\
        u(1)=1
    \end{cases}	
\end{equation}
\subsection{精确解}
将原方程化为$tdu+udt=0$,则有$d(ut)=0$,解得$ut=C$($C$为常数),代入初始条件得$C=1$,从而该方程的精确解为:
\begin{displaymath}
    u=\frac{1}{t},(1\leq t \leq2).
\end{displaymath}
\subsection{欧拉方法}
代入欧拉格式得:
\begin{displaymath}
    u_{i+1}=u_{i}+hf(t_i,u_i)=u_i+h(-\frac{u_i}{t_i})
\end{displaymath}
\subsection{隐式欧拉方法}
由隐式欧拉格式得:
\begin{displaymath}
    u_{i+1}=u_{i}+hf(t_{i+1},u_{i+1})=u_i+h(-\frac{u_{i+1}}{t_{i+1}}),
\end{displaymath}
移项化简可得:
\begin{displaymath}
    u_{i+1}=\frac{t_{i+1}u_i}{t_{i+1}+h}
\end{displaymath}

\subsection{程序}
根据上述推导,用python编写程序,代码如下:
\begin{lstlisting}
# implict euler method
import numpy as np
import matplotlib.pyplot as plt

# the right term of the ODE
def f(t, u):
    f = -u/t
    return f

# the exact solution of ODE 
def fexact(t):
    fexact = 1/t
    return fexact

N = 100
t_n = 2.0
dt = (t_n - 1.0) / N
t = np.arange(1.0, t_n + dt, dt)
u_euler = np.arange(1.0, t_n + dt, dt)
u = np.arange(1.0, t_n + dt, dt)
u_true = np.arange(1.0, t_n + dt, dt)

i = 0
while i < N:
    t[i+1] = t[i] + dt
    u_euler[i+1] = u_euler[i] + dt * f(t[i], u_euler[i])
    u[i+1] = (u[i] * t[i+1])/(t[i+1] + dt)
    u_true[i+1] = fexact(t[i+1])
    i = i + 1

err_euler = max(abs(u_euler - u_true))
err_implict_euler = max(abs(u - u_true))
print("The error of euler method: ",err_euler)
print("The error of implict euler method: ",err_implict_euler)

# begin drawing
plt.title('Result')
plt.plot(t, u_euler, color='green', label='euler')
plt.plot(t, u, color='blue', label='implict euler')
plt.plot(t, u_true, color='red', label='exact')
plt.legend() # show the legend

plt.xlabel('t')
plt.ylabel('u')
plt.show()
\end{lstlisting}

\subsection{结果分析}
当取$h=0.01$时,此时欧拉方法的误差为0.02631578947368396,隐式欧拉方法的误差为0.023809523809523836,结果如下图所示:

当取不同$h$,得到的误差如下表所示:

\begin{longtable}{ccccc}
    \caption{不同$h$的误差表}\\\hline
    $h$ & 欧拉方法 & 
    \multicolumn{1}{c}{隐式欧拉方法} \\\hline
    \endfirsthead
    \caption[]{不同$h$的误差表(续表)}\\
    \multicolumn{3}{r}{\footnotesize 接上页}\\\hline
    $h$ & 欧拉方法 & \multicolumn{1}{c}{隐式欧拉方法}\\
    \hline\endhead
    \hline\multicolumn{3}{r}{\footnotesize 接下页}\\
    \endfoot\hline\hline\endlastfoot
    $\frac{1}{2}$ & 0.16666666666666663 & 0.09999999999999998   \\
    $\frac{1}{4}$ & 0.0714285714285714 & 0.05555555555555558   \\
    $\frac{1}{8}$ & 0.033333333333333215 & 0.02941176470588236   \\
    $\frac{1}{16}$ & 0.01612903225806467 & 0.015151515151515138   \\
    $\frac{1}{32}$ & 0.00793650793650813 & 0.007692307692307665   \\
    $\frac{1}{64}$ & 0.003937007874015519 & 0.003875968992248069   \\
    $10^{-1}$ & 0.02631578947368396 & 0.023809523809523836    \\
    $10^{-2}$ & 0.0025125628140699185 & 0.0024875621890547706   \\
    $10^{-3}$ & 0.0002501250625697726 & 0.0002498750624687629   \\
\end{longtable}


\end{document}